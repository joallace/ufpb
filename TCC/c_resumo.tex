\section*{\centering{RESUMO}}
% Um resumo de trabalho de conclusão de curso é do tipo informativo e deve conter somente um parágrafo. A estrutura do resumo deve conter essencialmente os seguintes tópicos: apresentar inicialmente os objetivos do trabalho (o que foi feito?), a justificativa (porquê foi feito) e, finalmente, os resultados alcançados. O resumo deve informar ao leitor todas as informações importantes para o que o leitor possa entender o trabalho desenvolvido, quais foram as finalidades, a metodologia que o autor utilizou e os resultados obtidos. Deve conter frases curtas, porém completas (evitar estilo telegráfico); usar o tempo verbal no passado para os principais resultados e presente para comentários ou para salientar implicações significativas.  O resumo em português e inglês são obrigatórios e não devem passar de 200 palavras.

% Redes sociais são plataformas online que visam conectar usuários, permitindo-os compartilhar conteúdos e interagir entre si.
% O presente relatório técnico descreve o processo de projeto, implementação e validação da aplicação de código aberto colcom, a qual tem como intuito principal ser um ambiente favorável para a participação e deliberação de indivíduos de uma certa comunidade em seus processos de decisão, promovendo assim a democracia digital. A plataforma utiliza de práticas de redes sociais e conceitos de Sistemas de Controle de Versão (SCV), particularmente o \textit{Git}, como principais diretrizes de funcionamento.

% \noindent xxxxxxxx O presente relatório técnico descreve o processo de projeto, implementação e validação da aplicação de código aberto colcom. Seu intuito principal é ser um ambiente favorável para a participação e deliberação de indivíduos de uma certa comunidade em seus processos de decisão, promovendo, assim, a democracia digital. A plataforma utiliza práticas de redes sociais e conceitos de Sistemas de Controle de Versão (SCV), particularmente o \textit{Git}, como principais diretrizes de funcionamento.


% \noindent XXXXX 
\noindent Ferramentas de participação digital já são uma realidade no Brasil e em outros países, contudo, seu acesso ainda se restringe a uma pequena parcela dos usuários de internet; redes sociais por outro lado já são bem difundidas, mas sofrem de limitações significativas no que tange à deliberação eficaz. O objetivo principal da plataforma aqui proposta é ser um ambiente favorável para a participação e deliberação de indivíduos de uma certa comunidade em seus processos de decisão, promovendo, assim, a democracia digital em escala local. O presente relatório técnico descreve o processo de projeto, implementação e validação da aplicação de código aberto \textit{colcom}. A plataforma utiliza dinâmicas de redes sociais e conceitos de Sistemas de Controle de Versão (SCV), particularmente o \textit{Git}, como principais diretrizes de funcionamento.

{\bf Palavras-chave:} deliberação, democracia digital,  redes sociais, participação social, aplicação web.

