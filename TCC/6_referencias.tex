% As referências devem seguir o padrão da ABNT.


\bibitem{polis}
SMALL, Christopher et al. Polis: Scaling deliberation by mapping high dimensional opinion spaces. \textbf{Recerca: revista de pensament i anàlisi}, v. 26, n. 2, 2021. Disponível em: http://dx.doi.org/10.6035/recerca.5516.

\bibitem{gomes2005democracia}
GOMES, Wilson. A democracia digital e o problema da participação civil na decisão política. \textbf{Revista Fronteiras}, v. 7, n. 3, p. 214-222, 2005.

\bibitem{reviewSocialMedia}
AICHNER, Thomas et al. Twenty-five years of social media: a review of social media applications and definitions from 1994 to 2019. \textbf{Cyberpsychology, behavior, and social networking}, v. 24, n. 4, p. 215-222, 2021. Disponível em: http://dx.doi.org/10.1089/cyber.2020.0134. 

\bibitem{arabSpring}
KHONDKER, Habibul Haque. Role of the new media in the Arab Spring. \textbf{Globalizations}, v. 8, n. 5, p. 675-679, 2011. Disponível em: http://dx.doi.org/10.1080/14747731.2011.621287. 

\bibitem{journey2013}
MACHADO, Jorge; MISKOLCI, Richard. Das jornadas de junho à cruzada moral: o papel das redes sociais na polarização política brasileira. \textbf{Sociologia \& Antropologia}, v. 9, p. 945-970, 2019. Disponível em: http://dx.doi.org/10.1590/2238-38752019v9310. 

\bibitem{euromaidan}
BOHDANOVA, Tetyana. Unexpected revolution: the role of social media in Ukraine's Euromaidan uprising. \textbf{European View}, v. 13, n. 1, p. 133-142, 2014. Disponível em: https://doi.org/10.1007/s12290-014-0296-4

\bibitem{blackLivesMatter}
MUNDT, Marcia; ROSS, Karen; BURNETT, Charla M. Scaling social movements through social media: The case of Black Lives Matter. \textbf{Social media+ society}, v. 4, n. 4, p. 2056305118807911, 2018. Disponível em: http://dx.doi.org/10.1177/2056305118807911. 

\bibitem{polarizationSocialMedia}
KUBIN, Emily; VON SIKORSKI, Christian. The role of (social) media in political polarization: a systematic review. \textbf{Annals of the International Communication Association}, v. 45, n. 3, p. 188-206, 2021. Disponível em: http://dx.doi.org/10.1080/23808985.2021.1976070. 

\bibitem{echoChamber}
CINELLI, Matteo et al. The echo chamber effect on social media. \textbf{Proceedings of the National Academy of Sciences}, v. 118, n. 9, p. e2023301118, 2021. Disponível em: http://dx.doi.org/10.1073/pnas.2023301118. 

\bibitem{ddBrazil} SAMPAIO, Rafael Cardoso et al. \textbf{Democracia digital no Brasil: mapeamento e análises de artigos publicados em periódicos entre 1999-2018}. 2021. Disponível em: http://dx.doi.org/10.38116/bapi25art2.

\bibitem{ddWorld} CONGGE, Umar et al. Digital democracy: A systematic literature review. \textbf{Frontiers in Political Science}, v. 5, p. 972802, 2023. Disponível em: https://doi.org/10.3389/fpos.2023.972802.

\bibitem{proGit} CHACON, Scott; STRAUB, Ben. \textbf{Pro git}. Springer Nature, 2014. Disponível em: https://doi.org/10.1007/978-1-4842-0076-6.

\bibitem{backFront} ABDULLAH, Hanin M.; ZEKI, Ahmed M. Frontend and backend web technologies in social networking sites: Facebook as an example. In: \textbf{2014 3rd international conference on advanced computer science applications and technologies}. IEEE, 2014. p. 85-89. Disponível em: https://doi.org/10.1109/ACSAT.2014.22.

\bibitem{database} SILBERSCHATZ, Abraham; KORTH, Henry F.; SUDARSHAN, Shashank. Database system concepts. 2011.

\bibitem{scv} ZOLKIFLI, Nazatul Nurlisa; NGAH, Amir; DERAMAN, Aziz. Version control system: A review. \textbf{Procedia Computer Science}, v. 135, p. 408-415, 2018. Disponível em: https://doi.org/10.1016/j.procs.2018.08.191.

\bibitem{requirements} GLINZ, Martin. On non-functional requirements. In: \textbf{15th IEEE international requirements engineering conference (RE 2007)}. IEEE, 2007. p. 21-26. Disponível em: https://doi.org/10.1109/RE.2007.45.

\bibitem{requirements2} CHUNG, Lawrence; DO PRADO LEITE, Julio Cesar Sampaio. On non-functional requirements in software engineering. \textbf{Conceptual modeling: Foundations and applications: Essays in honor of john mylopoulos}, p. 363-379, 2009. Disponível em: https://doi.org/10.1007/978-3-642-02463-4\_19

\bibitem{friction} TSENG, Yu-Shan. Rethinking gamified democracy as frictional: a comparative examination of the Decide Madrid and vTaiwan platforms. \textbf{Social \& Cultural Geography}, v. 24, n. 8, p. 1324-1341, 2023. Disponível em: http://dx.doi.org/10.1080/14649365.2022.2055779.

\bibitem{scienceGit} RAM, Karthik. Git can facilitate greater reproducibility and increased transparency in science. \textbf{Source code for biology and medicine}, v. 8, p. 1-8, 2013. Disponível em https://doi.org/10.1186/1751-0473-8-7.

\bibitem{blueInconclusive} WONG, Nikita A.; BAHMANI, Hamed. A review of the current state of research on artificial blue light safety as it applies to digital devices. \textbf{Heliyon}, v. 8, n. 8, 2022. Disponível em: https://doi.org/10.1016/j.heliyon.2022.e10282.

\bibitem{blueWorsenSleep} PHIPPS-NELSON, Jo et al. Blue light exposure reduces objective measures of sleepiness during prolonged nighttime performance testing. \textbf{Chronobiology International}, v. 26, n. 5, p. 891-912, 2009. Disponível em: https://doi.org/10.1080/07420520903044364

\bibitem{socialMediaBots} ORABI, Mariam et al. Detection of bots in social media: a systematic review. \textbf{Information Processing \& Management}, v. 57, n. 4, p. 102250, 2020. Disponível em: https://doi.org/10.1016/j.ipm.2020.102250.


% \bibitem{vTaiwan}
% vTaiwan: rethinking democracy. vTaiwan, 2024. Disponível em: https://info.vtaiwan.tw/. Acesso em: 25 de março de 2024.

\bibitem{brasilParticipativo}
Sobre o Brasil Participativo. gov.br, 2023. Disponível em: https://brasilparticipativo.presidencia.gov.br/processes/brasilparticipativo/f/33/. Acesso em: 25 de março de 2024.

\bibitem{comInternet}
Usuários de Internet - Indicador Ampliado. cetic, 2023. Disponível em: https://cetic.br/pt/tics/domicilios/2023/individuos/C2A/. Acesso em: 25 de março de 2024.

\bibitem{globalSocialMediaStatistics}
Global social media statistics. DataReportal, 2024. Disponível em: https://datareportal.com/social-media-users/. Acesso em: 24 de março de 2024.

\bibitem{brazilSocialMediaStatistics}
Digital 2024: Brazil. DataReportal, 2024. Disponível em: https://datareportal.com/reports/digital-2024-brazil/. Acesso em: 3 de abril de 2024.


% %para livro%
% \bibitem{livro} SOBRENOME, Nome. {\bf Título do livro em negrito}. Cidade: Editora, ano.

% %para revista científica%

% \bibitem{revista}SOBRENOME, Nome. Título do artigo. {\bf Nome da revista em negrito}, volume, número, páginas, mês, ano.

% %para anais de evento em meio eletrônico%
% \bibitem{anais}SOBRENOME, Nome. Título do artigo. In: Nome do evento, Edição, Local do evento. {\bf Anais eletrônicos...} Entidade patrocinadora do evento: Editora|, ano. CD-ROM.

% %para capítulo de livro%

% \bibitem{caplivro}SOBRENOME, Nome. {\bf Título do capítulo}. In: Responsável pela organização do livro (Org.). Título do livro. Cidade: Editora, ano.


% %para dissertação ou tese%
% \bibitem{dissertacao}SOBRENOME, Nome. {\bf Título}: subtítulo. ano. Dissertação (ou Tese) – Departamento acadêmico, Universidade, Cidade, ano.

% %Internet%
% \bibitem{internet}SOBRENOME, Nome. Título. Cidade: Organização, ano. Disponível em: $<$http://***$>$. Acesso em: dia (não incluir o zero à esquerda) mês (usar abreviações) ano.

