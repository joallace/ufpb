%Abstract%
\section*{\centering{ABSTRACT}} 
\noindent Digital participation tools are already a reality in Brazil and other countries, however, their access is still restricted to a small portion of internet users; Social networks, on the other hand, are already widespread, but suffer from significant limitations in terms of effective deliberation. The main objective of the platform here proposed is to be a favorable environment for the participation and deliberation of individuals from a certain community in their decision-making processes, thus promoting digital democracy on a local scale. This technical report describes the design, implementation and validation process of the \textit{colcom} open source application. The platform uses social network dynamics and Version Control Systems (VCS) concepts, particularly \textit{Git}, as its main operating guidelines.

{\bf Key-words:} deliberation, digital democracy, social networks, social participation, web application.