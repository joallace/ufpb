\section{INTRODUÇÃO}

A democracia digital é um conceito que começou a ser formulado após o advento e expansão da internet — que ocorreu concomitantemente à popularização de dispositivos que fazem uso dela, como computadores pessoais e aparelhos celulares —  e que vem se modernizando e concretizando juntamente com o desenvolvimento tecnológico. Valendo-se desta difusão do acesso à internet, foram propostas, nacionalmente e internacionalmente, diversas ferramentas que visam expandir a prática da democracia.

No cenário internacional, um dos maiores exemplos é o sistema \textit{Pol.is}\footnote{Disponível em: https://pol.is/. Acesso em: 25 de março de 2024.} \cite{polis} que já foi utilizado em 9 países diferentes, sendo o seu caso de uso de maior destaque em Taiwan, com o processo \textit{vTaiwan}\footnote{Disponível em: https://info.vtaiwan.tw/. Acesso em: 25 de março de 2024.}, iniciado em 2014, em que, de acordo com seu próprio \textit{website}, já foram discutidas mais de 28 pautas por mais de 200 mil participantes. 80\% dessas discussões levaram à alguma ação governamental decisiva.

No cenário nacional, o projeto de maior destaque é a plataforma \textit{Brasil Participativo}\footnote{Disponível em: https://brasilparticipativo.presidencia.gov.br/. Acesso em: 25 de março de 2024.}, que tem o título de maior experiência de participação social na internet já realizada pelo governo federal \cite{brasilParticipativo}. O projeto contou com mais de 1,4 milhões de participantes em sua primeira iniciativa digital de participação, o Plano Plurianual Participativo 2024 - 2027, em que os ministérios tiveram a responsabilidade de analisar os conteúdos propostos. Em seguida, o Ministério do Planejamento e Orçamento elaborou um documento de devolutiva contendo o que foi incorporado e a justificativa do que não foi incorporado, e por fim, o plano foi instituído na lei 14.802/2024\footnote{Disponível em: http://planalto.gov.br/ccivil\_03/\_ato2023-2026/2024/lei/L14802.htm. Acesso em: 1 abril de 2024.}, sancionada pelo presidente Lula. 

Entretanto, ainda há uma grande defasagem na adoção destas ferramentas pela população em geral, angariando somente uma pequena parcela dos usuários da internet, que ainda sequer constituem a totalidade da população. Considerando dados de 2023, cerca de 88\% da população brasileira tinha acesso à internet \cite{comInternet}, i.e., aproximadamente 178 milhões de pessoas, de modo que menos de 1\% dos usuários de internet brasileiros foram de fato ouvidos na maior experiência de participação social online do Brasil. Afinal, como enfatizado por Gomes (2005) \cite{gomes2005democracia}: \textquotedblleft[...] talvez nem toda a debilidade de participação política contemporânea se explique em termos de dificuldade de acesso, raridade de meios e escassez de oportunidades. A abundância de meios e chances não formará, per se, uma cultura da participação política.\textquotedblright.

Por outro lado, redes sociais já são uma forma de comunicação bem estabelecida no mundo, contando com cerca de 5 bilhões de contas ativos globalmente no começo de 2024 \cite{globalSocialMediaStatistics}, das quais 144 milhões são brasileiras \cite{brazilSocialMediaStatistics}, o que corresponde em torno de 81\% de todos os usuários de internet do país. Apesar das maiores aplicações das mídias sociais não serem voltadas necessariamente para a ação política \cite{reviewSocialMedia}, o seu impacto neste setor já é extensivamente documentado pelo mundo, como por exemplo: durante a Primavera Árabe \cite{arabSpring}, nas Jornadas de Junho de 2013 \cite{journey2013}, no \textit{Euromaidan}\cite{euromaidan} e no movimento \textit{Black Lives Matter} \cite{blackLivesMatter}.
% o que corresponde à aproximadamente 62\% da população mundial neste período. 

\subsection{Definição do Problema}
As ferramentas de participação social na política ainda não conquistaram a adesão de uma porção significativa dos utilizadores de internet, e as redes sociais existentes, apesar de populares, ainda são ferramentas limitadas quando o assunto é democracia e prática política. Já existem diversas pesquisas que apontam os efeitos de polarização \cite{journey2013, polarizationSocialMedia} e de agrupamento de indivíduos com ideias semelhantes \cite{echoChamber}, fenômeno também conhecido como \textit{echo chambers}, nas redes mais utilizadas. Estes fatos se relacionam profundamente com a falta de transparência algorítmica, por parte das empresas que desenvolvem tais plataformas, de maneira que conteúdos são fornecidos aos usuários com base em critérios não claros, ou sequer divulgados.

\subsection{Premissas e Hipóteses}

Levando em conta a cultura já estabelecida de uso das redes sociais, da atenção crescente que o tema da democracia digital vem recebendo no meio acadêmico \cite{ddBrazil, ddWorld} e da crescente, mas ainda tímida, quantidade de aplicações que buscam incrementar o papel da participação popular no governo, observamos uma possibilidade de contribuir para o tema.

A hipótese central para o desenvolvimento desta aplicação  baseia-se no sucesso do uso de Sistemas de Controle de Versão (SCV), particularmente o \textit{Git}. O \textit{Git} é mais comumente utilizado no contexto de desenvolvimento de software, principalmente de maneira colaborativa, entretanto, essa não é a única aplicação para o \textit{Git}, sendo o mesmo já usado como uma ferramenta para auxiliar na produção de conhecimento científico \cite{scienceGit}, mostrando que há uma maior gama de aplicações possíveis com o uso do sistema. Assim, visamos aplicar conceitos de SCV para a construção coletiva de ideias, de forma a potencializar a capacidade de indivíduos de uma certa comunidade chegarem à soluções mais bem desenvolvidas e com participação de todos os seus membros.

% A melhor forma de determinar o tema abordado é através de premissas e hipóteses. A hipótese consiste em uma afirmativa que você considera verdadeira e que vai provar ou buscar provar ao longo de seu trabalho. Outra forma é delimitando o problema em forma de uma pergunta de partida. As hipóteses apresentadas aqui são provadas no seu trabalho é o que chamoas de tese.

\subsection{Objetivo geral}
Temos como objetivo projetar e implementar uma rede social online de código aberto, acessível por computadores e dispositivos móveis, que facilite e aprimore o processo de participação de integrantes de uma dada comunidade em seus processos de decisão.

% \subsection{Objetivos específicos}
% \begin{itemize} 
%     \item Identificar e especificar requisitos funcionais e não funcionais da aplicação;
%     \item Descrever as tecnologias utilizadas;
%     \item Desenvolver uma plataforma baseada em soluções bem estabelecidas.
% \end{itemize}

% XXXX

\subsection{Objetivos específicos}
\begin{itemize} 
    \item Implementar um banco de dados integrando \textit{Git} e \textit{PostgreSQL};
    \item Desenvolver uma interface de fácil compreensão e com soluções bem estabelecidas;
    % \item Fazer um teste com usuários reais e aplicar um questionário;
    \item Fazer um comparativo com plataformas semelhantes;
    \item Disponibilizar o código fonte em um repositório no \textit{Github};
\end{itemize}

\subsection{Estrutura do relatório técnico}
O presente relatório está dividido em 5 capítulos. No primeiro, é realizada a introdução ao tema e a definição do problema e dos objetivos do trabalho. No segundo, são expostos os principais conceitos e tecnologias utilizados para a construção da solução. No terceiro, são desenvolvidos os requisitos da aplicação proposta e a sua arquitetura. No quarto, são apresentados o plano de testes, um comparativo com plataformas semelhantes e uma análise sobre os resultados. No quinto e último capítulo são disponibilizadas as conclusões obtidas sobre o trabalho, bem como uma lista de sugestões para trabalhos futuros.