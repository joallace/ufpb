\section*{\centering{AGRADECIMENTOS}} 
Primeiramente, gostaria de agradecer a meus pais, José Wallace e Ivônia, pois sem seu apoio e influência não seria quem sou hoje e não poderia sequer sonhar em conquistar o que conquistei.

Em seguida, agradeço meu irmão Chrístian por ser uma companhia sempre presente em minha vida, e por ser sempre o primeiro a testar os limites das minhas ideias.

Não posso deixar de agradecer aos meus outros familiares, tios, tias, primos, primas, avós e avôs falecidos, e inclusive aos meus amigos — os quais considero minha segunda família —, cada um deles que me marcou e me moldou de alguma forma.

Também agradeço a Hevelyn Mafra, por ter sido uma namorada amorosa e companheira, sendo um grande apoio para mim durante a produção deste trabalho e de boa parte de minha graduação.

Agradeço ao meu orientador Dr. Yuri Malheiros, pela disponibilidade e solicitude, o que me habilitou a fazer este trabalho com segurança. Ademais, também o agradeço por ter sido um supervisor prestativo durante os projetos em que trabalhamos juntos.

Agradeço a minha professora e examinadora Dra. Thaís Gaudêncio, que, além de me ajudar enormemente na formulação do presente trabalho e a lidar com as burocracias com o processo de conclusão de curso, foi a primeira a me cativar para a área e acompanhou boa parte de minha trajetória na graduação;

Agradeço ao Dr. Alexandre Duarte, que prontamente aceitou o convite para participar da banca.

Agradeço a meus outros professores e mentores durante o curso, Anand Subramanian, que me aconselhou extensamente durante minha graduação; Marcelo Iury e Telmo Filho, que me guiaram muitas vezes em minhas experiências profissionais.

Agradeço, também, aos meus amigos que conheci durante minha graduação, principalmente João Pedro e Itamar, os quais foram influências muito positivas para meu desenvolvimento no curso e tornaram o peso do estudo um pouco menor.

Por fim, agradeço a todos que não me auxiliaram diretamente, mas que sem sua contribuição meu trabalho não seria possível: todos os trabalhadores da UFPB, todos os desenvolvedores que divulgam seus códigos e promovem o cenário \textit{open-source} e todos aqueles que visaram ampliar o papel do povo nas decisões políticas.